\documentclass{beamer}
\usepackage[latin1]{inputenc}
\usepackage{minted}

\usetheme{default}
\usecolortheme{dove}

%gets rid of bottom navigation bars
\setbeamertemplate{footline}[page number]{}

%gets rid of navigation symbols
\setbeamertemplate{navigation symbols}{}

% center titles
\setbeamertemplate{frametitle}[default][center]

\title[Monads]{\Huge An Introduction to Monads}
\author{Phillip Mates}
\begin{document}

\begin{frame}
\titlepage
\end{frame}

\begin{frame}{Why Monads?}

 In a purely functional language:
  \begin{itemize}
    \item How do you encode actions with side-effects, such as reading and writing files?
    \item Is there an elegant way to pass around program state without explicitly threading it in and out of every function?
    \item How do you code up doubly nested for-loops?
    \item What about: Continuation passing style, Writing logs, Memory transactions\ldots
  \end{itemize}
\end{frame}

\begin{frame}{What are Monads?}
  They're a very general abstraction idea that can be thought of as:
  \vspace{.25in}
  \begin{itemize}
    \item containers that wrap values and are composable
    \item the inverse of pointers
    \item an abstraction for modeling sequential actions
    \item \ldots
  \end{itemize}
\end{frame}

\defverbatim[colored]\maybeType{
  \begin{minted}{haskell}
data Maybe a = Nothing
             | Just a

lookup :: a -> [(a, b)] -> Maybe b

animalFriends :: [(String, String)]
animalFriends = [ ("Pony", "Lion")
                , ("Lion", "Manticore")
                , ("Unicorn", "Lepricon") ]
  \end{minted}
}

\begin{frame}{Error handling with Maybe}
  \maybeType
\end{frame}

\defverbatim[colored]\maybeChain{
  \begin{minted}{haskell}
-- Does Pony's friend have a friend in animalMap?
animalFriendLookup :: [(String, String)] -> Maybe String
animalFriendLookup animalMap =
  case lookup "Pony" animalMap of
       Nothing -> Nothing
       Just ponyFriend ->
         case lookup ponyFriend animalMap of
              Nothing -> Nothing
              Just ponyFriendFriend ->
                case lookup ponyFriendFriend animalMap of
                     Nothing -> Nothing
                     Just friend -> Just friend
  \end{minted}
}

\begin{frame}
  \maybeChain
\end{frame}


\defverbatim[colored]\monad{
  \begin{minted}{haskell}
-- Bind
(>>=) :: m a -> (a -> m b) -> m b

-- Inject value into a container
return :: a -> m a
  \end{minted}
}

\begin{frame}{Monads are comprised of two functions}
  \monad
\end{frame}


\defverbatim[colored]\maybeDef{
  \begin{minted}{haskell}
-- (>>=) :: m a -> (a -> m b) -> m b
Just x >>= k  =  k x
Nothing >>= _ =  Nothing

-- return :: a -> m a
return x      =  Just x
  \end{minted}
}

\begin{frame}{Maybe Monad}
  \maybeDef
\end{frame}

\defverbatim[colored]\maybeBind{
  \begin{minted}{haskell}
monadicFriendLookup :: [(String, String)] -> Maybe String
monadicFriendLookup animalMap =
  lookup "Pony" animalMap
  >>= (\ponyFriend -> lookup ponyFriend animalMap
  >>= (\pony2ndFriend -> lookup pony2ndFriend animalMap
  >>= (\friend -> Just friend)))
  \end{minted}
}

\begin{frame}{Using Maybe as a Monad}
  \maybeBind
\end{frame}

\defverbatim[colored]\maybeDo{
  \begin{minted}{haskell}
-- or even better:
sugaryFriendLookup :: [(String, String)] -> Maybe String
sugaryFriendLookup animalMap = do
  ponyFriend   <- lookup "Pony" animalMap
  ponyFriend'  <- lookup ponyFriend animalMap
  ponyFriend'' <- lookup ponyFriend' animalMap
  return friend
  \end{minted}
}

\begin{frame}{Using Maybe as a Monad}
  \maybeDo
\end{frame}

\defverbatim[colored]\transformStmt{
  \begin{minted}{haskell}
type Sexpr = String

-- naive generation of unique symbol
transformStmt :: Sexpr -> Int -> (Sexpr, Int)
transformStmt expr counter = (newExpr, counter+1)
  where newExpr = "(define " ++ var ++ " " ++ expr ++ ")"
        var = "tmpVar" ++ (show counter)
  \end{minted}
}

\begin{frame}{Threading program state}
  \transformStmt
\end{frame}

\defverbatim[colored]\chop{
  \begin{minted}{haskell}
Int -> (Sexpr, Int)
  \end{minted}
}

\defverbatim[colored]\transType{
  \begin{minted}{haskell}
transformStmt :: Sexpr -> Int -> (Sexpr, Int)
  \end{minted}
}

\defverbatim[colored]\transState{
  \begin{minted}{haskell}
transformStmt :: Sexpr -> State Int Sexpr
  \end{minted}
}
\defverbatim[colored]\generalizedState{
  \begin{minted}{haskell}
newtype State s a = State {
      runState :: s -> (a, s)
    }
  \end{minted}
}

\begin{frame}[fragile, containsverbatim]{Generalizing the threading of state}
  Let's drop \\
  \chop
  from \\
  \transType
  and replace it with a more general type constructor:
  \vspace{.25in}
  \uncover<2->{\generalizedState}
  \uncover<2->{\transState}
\end{frame}

\defverbatim[colored]\stateDef{
  \begin{minted}{haskell}
-- return :: a -> State s a
return a = State (\s -> (a, s))

-- (>>=) :: State s a -> (a -> State s b) -> State s b
m >>= k = State (\s -> let (a, s') = runState m s
                      in runState (k a) s')
  \end{minted}
}

\begin{frame}{State Monad}
  \stateDef
\end{frame}

\begin{frame}{State Monad Example}
\end{frame}


\defverbatim[colored]\stateTypeCons{
  \begin{minted}{haskell}
-- this can't because it has arity 2:
ghci> :kind State
* -> * -> *

-- but these have arity 1:
ghci> :kind (State Int)
* -> *

ghci> :kind []
* -> *
  \end{minted}
}

\begin{frame}{What can be a Monad?}
  Type constructors with an arity of one, for instance:
  \stateTypeCons
\end{frame}

\defverbatim[colored]\typeDirectedList{
  \begin{minted}{haskell}
ghci> :type (>>=)
(>>=) :: (Monad m) => m a -> (a -> m b) -> m b

ghci> :type map
map :: (a -> b) -> [a] -> [b]

ghci> :type flip map
flip map :: [a] -> (a -> b) -> [b]

ghci> :type concat
concat :: [[a]] -> [a]
  \end{minted}
}

\begin{frame}{Deriving the list monad}
  \typeDirectedList
\end{frame}

\defverbatim[colored]\listMonad{
  \begin{minted}{haskell}
return x = [x]
xs >>= f = concat (map f xs)
  \end{minted}
}

\defverbatim[colored]\powerset{
  \begin{minted}{haskell}
-- monadic powerset
ghci> powerset = [1,2]
             >>= (\i -> [1..4]
             >>= (\j -> [(i, j)]))
[(1,1),(1,2),(1,3),(1,4),(2,1),(2,2),(2,3),(2,4)]
  \end{minted}
}

\begin{frame}[fragile]{The List monad models non-determinism}
\listMonad
\uncover<2->{\powerset}
\end{frame}


\defverbatim[colored]\desugar{
  \begin{minted}{haskell}
do x <- foo     ===     foo >>= (\x -> bar)
   bar

do act1         ===     act1 >> act2
   act2
  \end{minted}
}

\begin{frame}{Desugaring do Blocks}
  \desugar
\end{frame}

\begin{frame}{Further Topics \& Reading}
  \begin{itemize}
    \item Monad Transformers
    \item ``Real World Haskell" by O'Sullivan, Stewart, and Goerzen
    \item Corresponding blog post: \url{quined.net/articles/monads.html}
  \end{itemize}
\end{frame}




\end{document}
